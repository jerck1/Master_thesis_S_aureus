\chapter{Estado del Arte}
Los resultados presentados a continuaci\'{o}n se obtienen mediante trabajo conjunto con el grupo COBO de la universidad de los Andes.\\

\section{Orientaci\'{o}n}
Si se define el \'{a}ngulo de orientaci\'{o}n de STX como el \'{a}ngulo m\'{a}s peque\~{n}o entre el vector normal a la membrana y el vector que recorre la cadena diaponeuresporenoica (o poli\'{e}nica), entonces el \'{a}ngulo es de 0 \textdegree cuando la cadena de STX se encuentre en una orientaci\'{o}n vertical mientras que el \'{a}ngulo ser\'{a} de 90 \textdegree si  la cadena de STX se ncuentra en una orientaci\'{o}n horizontal.\\
%resultados
%#orientaci\'{o}n
Las distribuciones del \'{a}ngulo de orientaci\'{o}n de STX se encuentran en la figura FIGURA. El \'{a}ngulo de orientaci\'{o}n para una sola estafiloxantina refleja un rango m\'{a}s corto de valores (distribuci\'{o}n m\'{a}s angosta) que para el caso de 15\% de STX, adem\'{a}s si se compara una STX inmersa en DMPG con una STX inmersa en DPPG se observa que el rango de posibles orientaciones de STX es mayor para el caso DMPG que para DPPG. De estas diferencias en los rangos se infiere que STX tiende a estar m\'{a}s vertical en presencia de DPPG que en presencia de DMPG. Otra caracter\'{i}stica que se encuentra en una de las distribuciones para una STX y no en la otra distribuci\'{o}n es el hecho para DMPG solo hay un m\'{a}ximo, mientras que para DPPG hay dos m\'{a}ximos, es decir DPPG es una distribuci\'{o}n bimodal. --Se pueden conjeturar varias cosas sobre la distribuci\'{o}n bimodal, por ejemplo que esta se obtiene porque alguna parte de la mol\'{e}cula se puede mover con mayor facilidad para el caso de DPPG, alguna parte como la cola de la cadena poli\'{e}nica de STX o el az\'{u}car de STX. Otra posibilidad es que exista alguna interacci\'{o}n que estabilice estas dos posibles orientaciones de STX.--\\

Para el caso de una sola mol\'{e}cula de STX y usando los par\'{a}metros obtenidos por optimizaci\'{o}n QM (STX r\'{i}gida) se observa que disminuye la amplitud del segundo pico en DPPG, lo cual nos indica que la aparici\'{o}n de un rango m\'{a}s amplio de orientaciones de STX en DPPG se deb\'{i}a a una mayor flexibilidad de la cadena poli\'{e}nica, con lo cual la mol\'{e}cula pod\'{i}a variar su longitud.\\

Para un 15\% de STX aparecen los dos picos tanto en el caso DMPG como en el caso DPPG y los rangos posibles son similares, pero la media para el caso de DMPG es mayor que para el caso de DPPG, esto nos indica que el \'{a}ngulo tiende a ser m\'{a}s vertical a un 15\% de STX inmersa en DPPG.\\ 

Por otro lado, en los diagramas de dispersi\'{o}n de la posici\'{o}n del az\'{u}car respecto al \'{a}ngulo de orientaci\'{o}n, se observa una ligera tendencia a la inserci\'{o}n del az\'{u}car dentro de la membrana a medida que aumenta el \'{a}ngulo de orientaci\'{o}n. Esta caracter\'{i}stica desaparece para el caso 15\% STX en DPPG, en donde no se observa una correlaci\'{o}n entre el \'{a}ngulo y la posici\'{o}n del az\'{u}car.\\
\section{Par\'{a}metro de Orden}
%#par\'{a}metro de orden
En la figura FIGURA se encuentra el par\'{a}metro de orden de los l\'{i}pidos para todos los sistemas: Figura a): cadena 1 DMPG, b) cadena 2 DMPG, c): cadena 1 DPPG, d) cadena 2 DPPG. El par\'{a}metro de orden de los l\'{i}pidos se incrementa significativamente  cuando la membrana contiene 15\% de STX llegando a ser este aumento de entre 0.02 a 0.04 cuando miramos los carbonos C4 a C6 de todos los casos, es decir, los l\'{i}pidos de la membrana est\'{a}n m\'{a}s organizados cuando hay mayor contenido de STX . Adem\'{a}s, si comparamos los casos de DMPG contra DPPG vemos que el par\'{a}metro de orden es menor en el orden de 0.01 entre los carbonos C4 a C6 para el caso DMPG que para el caso DPPG, entonces se puede ver que la membrana est\'{a} m\'{a}s ordenada cuando el 15\% de STX est\'{a} inmerso en DPPG. Otra diferencia se encuentra entre las cadenas 1 y 2 para cada caso. En ambos sistemas la cadena 2 presenta un valle en el \'{a}tomo C2, esto posiblemente se deba a que una de las cadenas es m\'{a}s larga que la otra.\\

En la figura FIGURA se muestra el par\'{a}metro de orden de STX para la cadena acil de STX a): DMPG, b): DPPG, se escogi\'{o} esta cadena ya que al ser m\'{a}s m\'{o}vil esperamos que sea m\'{a}s sensible a cambios en las propiedades de los par\'{a}metros de la mol\'{e}cula. Para los casos de una sola STX no hubo un incremento significativo del par\'{a}metro de orden pero para un 15\% increment\'{o} significativamente excepto para el primer \'{a}tomo de la cadena acil. Para una sola mol\'{e}cula.
\section{Espesor de la Membrana}
%espesor de la membrana

En la figura FIGURA, se muestra la densidad electr\'{o}nica normalizada respecto al eje z de la membrana (normal), las l\'{i}neas punteadas son las posiciones de los fosfatos para las bicapas puras. Tomando la las l\'{i}neas punteadas, el espesor de la membrana para DMPG es de 3.5nm, mientras que para DPPG es 3.8nm. La densidad electr\'{o}nica de la cadena acil de los l\'{i}pidos no es cero en el centro de la membrana, esto quiere decir que hay interdigitaci\'{o}n. El ox\'{i}geno O3,  presente en el az\'{u}car de STX, est\'{a} m\'{a}s inmerso en la membrana puesto que su posici\'{o}n es menor a la del fosfato de la derecha, esto quiere decir que el az\'{u}car est\'{a} hacia adentro de la membrana. En todos los casos la posici\'{o}n de este O3, en promedio, est\'{a} en el mismo lugar, sin embargo, en el caso de 15\% de STX. \\

El par\'{a}metro de orden est\'{a} relacionado con las constantes el\'{a}sticas del sistema REF DOULIEZ - ORDER PARAMETER.

\'{A}ngulo medio: 15STX-DMPG>15STX-DPPG \\
Espesor: DPPG=15STX-DPPG>15STX-DMPG=DMPG \\
APL: DMPG=DPPG>15STX-DMPG>15STX-DPPG \\



\section{Propiedades Globales de las Membranas Simuladas}
\subsection{Area por L\'{i}pido}
\section{Propiedades Locales de los Compuestos de la Membrana}
\subsection{\'{A}ngulos dihedros de la Estafiloxantina}
\subsection{Orientaci\'{o}n de la Estafiloxantina}
\subsection{Posici\'{o}n del az\'{u}car  de la Estafiloxantina y Puentes de Hidr\'{o}geno}
\subsection{Area Local por L\'{i}pido}