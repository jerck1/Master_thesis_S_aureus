\newpage
\textbf{\LARGE Resumen}
\addcontentsline{toc}{chapter}{\numberline{}Resumen}\\\\
\textit{Staphylococcus aureus} es un pat\'{o}geno de importancia cl\'{i}nica que ha recibido atenci\'{o}n p\'{u}blica debido al desarrollo de resistencia a diferentes antibi\'{o}ticos tradicionales. Los antibi\'{o}ticos tradicionales son principalmente activos en inhibir la s\'{i}ntesis de la pared de p\'{e}ptidoglicando que recubre la bacteria. Sin embargo, la bacteria ha mutado para generar resitencia a estos antibioticos. Como alternativa de tratamiento se est\'{a} estudiando el mecanismo de acci\'{o}n de diferentes p\'{e}ptidos antimicrobianos dirigidos a comprometer la integridad de la membrana lip\'{i}dica de la bacteria. Es importante estudiar las propiedades mecanicas de la membrana plasm\'{a}tica de la bacteria ya que la modulaci\'{o}n de estas propiedades, a trav\'{e}s de cambios en su composici\'{o}n lip\'{i}dica, puede resultar en incrementos en resistencia a estos p\'{e}ptidos antimicrobiales. Uno de los compuestos principales de la membrana de \textit{Staphylococcus aureus} es  el carotenoide estafiloxantina, el cual aumenta  su concentraci\'{o}n en la membrana plasm\'{a}tica cuando la bacteria est\'{a} sometida a estr\'{e}s. Se sugiere que este l\'{i}pido principalmente act\'{u}a como antioxidante que protege a la membrana de estr\'{e}s oxidativo. Algunos estudios han sugerido que otro papel de este carotenoide es aumentar la rigidez de la membrana plasm\'{a}tica de la bacteria. Sin embargo, a\'{u}n no est\'{a}n claras las propiedades biof\'{i}sicas locales que afecta la presencia de este carotenoide, tales como el efecto en el orden de las cadenas debido a la orientaci\'{o}n de la mol\'{e}cula dentro de la membrana y el efecto de su presencia en la constante de difusi\'{o}n. Uno de los m\'{e}todos que puede contribuir al estudio del papel de la estafiloxantina en la membrana es el de las simulaciones por din\'{a}mica molecular.\\
Debido al vac\'{i}o existente en el conocimiento del efecto de la estafiloxantina en membranas de \textit{Staphylococcus aureus}, aqu\'{i} se estudia el efecto de la presencia de estafiloxantina (STX) en sistemas sencillos de un componente mediante simulaciones por din\'{a}mica molecular, usando como l\'{i}pidos modelo DMPG y DPPG a una temperatura en donde los l\'{i}pidos se encuentran en fase l\'{i}quida-cristalina. La estafiloxantina se inserta en membranas modelo a las siguientes concentraciones: 1STX:128DMPG, 1STX:128DPPG, 15\% mol de STX en DMPG y 15\% mol STX en DPPG. Las simulaciones se realizan utilizando el campo de fuerza charmm36 para los l\'{i}pidos y se optimizan los par\'{a}metros de los \'{a}ngulos dihedros de los diferentes grupos moleculares de estafiloxantina para tratar de reflejar el comportamiento f\'{i}sico de la mol\'{e}cula. La optimizaci\'{o}n de los \'{a}ngulos se realiza en colaboraci\'{o}n con el grupo COBO dirigido por Gian Pietro Miscione del Departmento de Qu\'{i}mica de la Universidad de los Andes. Debido a que los par\'{a}metros de los \'{a}ngulos dihedros del grupo carbonil no cuentan con una parametrizacion adecuada, el grupo del Dr. Miscione realiza una optimizaci\'{o}n QM del potencial dih\'{e}drico de los \'{a}tomos de este grupo y de los enlaces dobles conjugados de la cadena diaponeurosporenoica. En la optimizaci\'{o}n se encuentran par\'{a}metros con una barrera energ\'{e}tica m\'{a}s alta, pero de forma s\'{i}mil, comparada con los par\'{a}metros originales. Con los par\'{a}metros optimizados se realizan nuevamente las simulaciones de los sistemas 1 STX en DMPG y 1 STX en DPPG.\\
De las simulaciones se obtienen propiedades biof\'{i}sicas como el par\'{a}metro de orden de las cadenas de los fosfol\'{i}pidos, la orientaci\'{o}n de la estafiloxantina (respecto a la normal de la membrana), el \'{a}rea por l\'{i}pido, el coeficiente de difusi\'{o}n, el perfil de presi\'{o}n, entre otras. En particular, una propiedad local como el perfil de presi\'{o}n (o perfil de tensi\'{o}n) proporciona una idea sobre los cambios en las interacciones repulsivas (presi\'{o}n positiva) y atractivas (presi\'{o}n negativa) presentes a cada nivel de profundidad de la membrana en presencia del carotenoide. 
El c\'{a}lculo de estas propiedades se realiza utilizando gromacs, mdstress y g\_lomepro.\\
Para el coeficiente de difusi\'{o}n y para los perfiles de presi\'{o}n se presentan los resultados de los cambios de estas cantidades en presencia de estafiloxantina.
En resultados preliminares de nuestro grupo se realizaron los diagramas de dispersi\'{o}n de la orientaci\'{o}n de estafiloxantina respecto a la posici\'{o}n del az\'{u}car y de la densidad electr\'{o}nica del ox\'{i}geno $\mathrm{O}_3$ (presente en el az\'{u}car de la estafiloxantina) Estos diagramas muestran que, para todos los sistemas analizados, el az\'{u}car tiende a insertarse dentro de la parte hidrof\'{o}bica de la membrana, lo cual se correlaciona con una tendencia a una posici\'{o}n m\'{a}s horizontal para la estafiloxantina. Por otro lado, las distribuciones de STX para una sola mol\'{e}cula y sin optimizar los par\'{a}metros muestran que el carotenoide puede estar en una conformaci\'{o}n cercana a la horizontal cuando est\'{a} en presencia de DMPG, pero que en el caso de DPPG puede tener conformaci\'{o}n horizontal y vertical (distribuci\'{o}n bimodal). Sin embargo, los casos en que la orientaci\'{o}n puede ser vertical dismuyen cuando se utilizan los par\'{a}metros optimizados. En cambio, para un 15\%mol STX, la estafiloxantina puede barrer las dos conformaciones, donde en promedio la estafiloxantina est\'{a} m\'{a}s horizontal para el caso de DMPG que para el caso de DPPG. Esto es acorde con el espesor de la membrana que es menor para DMPG que para DPPG, lo cual puede explicar la tendencia de la molecula de preferir orientaciones horizontales. Del par\'{a}metro de orden tanto para los l\'{i}pidos como para STX se puede ver que la membrana est\'{a} m\'{a}s ordenada cuando el 15\% de STX est\'{a} inmerso en DPPG, seguido de 15\% de STX en DMPG. El \'{a}rea por l\'{i}pido disminuye al agregar un 15\% de STX y con un valor mayor para DMPG que para DPPG. En conclusi\'{o}n, al agregar un 15\% de estafiloxantina, la membrana se vuelve m\'{a}s ordenada, m\'{a}s compacta en el plano lateral, pero sin cambiar significativamente su espesor. Esto se puede explicar con base en la interdigitaci\'{o}n de los carotenoides, de los l\'{i}pidos y por la capacidad de la estafiloxantina de muestrear diversas orientaciones, donde el rango de orientaciones es mayor en cuanto menor sea la longitud de la cadena acil lip\'{i}dica. Con base en estos resultados exploramos el comportamiento de la constante de difusi\'{o}n de los carotenoides con base en sus niveles de interdigitaci\'{o}n. Adicionalmente, exploramos los efectos en la constante de difusi\'{o}n de los fosfol\'{i}pidos debido a la presencia de estafiloxantina. La orientaci\'{o}n horizontal debe influir fuertemente en el perfil de presi\'{o}n de la membrana, asi que analizamos los efectos de la estafiloxantina en sobre este perfil.\\[4.0cm]
\textbf{\LARGE Abstract}\\\\
\textit{Staphylococcus aureus} is a clinically important pathogen that has received public attention because the bacterium has developed resistance to different antibiotics. These antibiotics inhibit the synthesis of the peptidoglycan wall that lines the bacteria. As a treatment alternative, the mechanism of action of antimicrobial peptides aimed at compromising the integrity of the bacterial lipid membrane is being studied. It is important to study the mechanical properties of the plasma membrane since the bacterium can modulate these properties through changes in its lipid composition, which can result in increases in resistance to antimicrobial peptides. One of the membrane lipid components that has been studied in \textit{Staphylococcus aureus} is the carotenoid staphyloxanthin, which increases its concentration in the plasma membrane when bacteria are subject to stress. In particular, staphyloxanthin has been associated to oxidative stress protection. However, some studies have suggested that the role of this carotenoid is to increase the stiffness of the bacterial plasma membrane, leading to resistance to mechanical stress and antimicrobial peptide activity. In spite of this physiologically important role, local biophysical properties of the carotenoid, such as orientation and interactions with nearby lipids, have not yet been studied in detail. With is in mind,oOne of the methods that can contribute to understanding the role of staphyloxanthin in the membrane is molecular dynamics simulations.\\
In this thesis work, staphyloxanthin is inserted into single componet model membranes composed of DMPG and DPPG and studied at a temperature where the lipid membrane is in the liquid-crystalline phase. We studied the following compositions: 1STX: 128DMPG, 1STX: 128DPPG, 15 \% mol STX in DMPG, 15 \% mol STX in DPPG. The simulations were performed using the charmm36 force field for lipids and the parameters of the dihedral angles of the different staphyloxanthin molecular groups are optimized to try to reflect the physical behavior of the molecule. Angle optimization was performed by the COBO group led by Gian Pitro Miscione from the Chemistry department at Universidad de los Andes. Because the dihedral angle  parameters present in carbonyl group do not have a proper parametrization, Miscione's group performed a QM optimization of the dihedral potential of the atoms of this group as well as the conjugated double bonds of the diaponeurosporenoic chain. After optimization, parameters with a higher energy barrier were found, making the molecule stiffer compared to the original parameters. With the optimized parameters, the simulations of the 1 STX systems in DMPG and 1 STX in DPPG were performed again. \\
Biophysical properties such as the deuterium order parameter, staphyloxanthine orientation (with respect to the normal of membrane), area per lipid, diffusion coefficient, pressure profile, among others were obtained from the simulations in collaboration with Miscione's group. In this thesis we also present results on local properties such as the pressure profile (or stress profile), which provides insight into the changes in repulsive (positive pressure) and attractive (negative pressure) interactions present at each depth level of the membrane in the presence of the carotenoid. Computation of these properties is done using gromacs, mdstress and g\_ lomepro. \\
In this thesis work, we complement the previous results with measurements of changes in diffusion coefficient of the phospholipids and carotenoid molecules and the pressure profiles of the membrane in the presence of staphyloxanthin. In previous results of our group, we have generated scatter diagrams of the orientation of the staphyloxanthin with respect to the position of the sugar and electronic density of oxygen $ \mathrm{O} _3$ of the staphyloxanthin sugar. These diagrams show that, for all of the systems analyzed, the sugar tends to insert into the hydrophobic part of the membrane, which correlates with a trend to a more horizontal position for staphyloxanthin. STX distributions for a single molecule and without optimizing the parameters show that the carotenoid can be in a conformation close to horizontal when it is in the presence of DMPG, but in the case of DPPG it can have both vertical and horizontal conformations (bimodal distribution ). However, the cases where the orientation can be vertical decrease when the optimized parameters (higher rigidity) are used. When the concentration is increased to 15\% mol STX, staphyloxanthin can sweep both conformations, where on average staphyloxanthin is more horizontal for DMPG than for DPPG. This is consistent with the thickness of the membrane which is less for DMPG than for DPPG, which pushes the coformation of the molecule to horizontal in the case of the thinner DMPG membrane. From the order parameter for both lipids and STX it can be seen that the membrane is more ordered when 15 \% of STX is immersed in DPPG, followed by 15 \% of STX in DMPG. The lipid area decreases with the addition of 15 \% STX and with a higher value for DMPG than for DPPG. In conclusion, by adding 15 \% staphyloxanthin, the membrane becomes more ordered, more compact in the lateral plane, but without significantly changing its thickness at the cost of lipid interdigitation. Based on this results, we expect the diffusion constants and pressure profiles to be strongly affected. In this thesis we seek the effects on the diffusion constant of phospholipids due to the presence of staphyloxanthin. Horizontal orientation should strongly influence profile pressure of the membrane, so we analyzed the effects of staphyloxanthin on this profile. \\[2.0cm]
\textbf{\small Keywords: Staphyloxanthin, \textit{Staphyloccocus aureus}, molecular dynamics, membrane biophysics, thickness membrane, orientation of carotenoids, pressure, diffusion coefficient.}\\
\\[2.0cm]
% \section*{Jurados}

% %Nombres de por lo menos 3 profesores que conozcan del tema. Uno de ellos debe ser profesor de planta de la Universidad de los Andes.

% \begin{itemize}
% \item Pilar Cossio (Universidad de Antioquia)\\
% \href{mailto: pilar.cossio@biophys.mpg.de
% }{ pilar.cossio@biophys.mpg.de
% }\\
% \href{https://www.biophys.mpg.de/en/cossio.html}{https://www.biophys.mpg.de/en/cossio.html}
% \item Juan M. Vanegas
%  (Universidad de Vermont)\\
% \href{mailto:juan.m.vanegas@gmail.com}{juan.m.vanegas@gmail.com}\\
% \href{http://vanegaslab.org/}{http://vanegaslab.org/}
% %\item Antonio Manu Forero Shelton (Universidad de los Andes)\\
% %\href{mailto:anforero@uniandes.edu.co}{anforero@uniandes.edu.co}
% \end{itemize}
% \section*{Firma del Director}
% %\vspace{1.5cm}
% \begin{figure}[h]
% \begin{center}
%     \includegraphics[scale=0.6]{firma.jpg}
%  % \caption{}
%   \label{fig:firma}
% \end{center}
% \end{figure}
