Los \'{a}ngulos dih\'{e}dricos de los enlaces dobles conjugados permanecen en la posici\'{o}n \textit{trans}.

El final de la cadena diaponurosporenoica puede muestrear m\'{a}s conformaciones pero tiene algunas conformaciones prohibidas debido impedimentos est\'{e}ricos de la estafiloxantina. %tiene la forma de un ``gancho" que
La estafiloxantina tiende a estar m\'{a}s horizontal en presencia de DMPG que en presencia de DPPG.

El \'{a}rea global por l\'{i}pido muestra disminuci\'{o}n en el valor medio sin cambios significativos dentro del error.

El APL local muestra una distribuci\'{o}n bimodal, en donde cada pico corresponde a dos especies: estafiloxantinas y l\'{i}pidos
APL de estafiloxantina es menor a la de los l\'{i}pidos debido a que el tama\~{n}o del az\'{u}car la cual es menor que la cabeza polar del l\'{i}pido
Esto marca diferencias entre las propiedades globales y locales, porque en las globales no aparecen las diferentes especies lip\'{i}dicas para el APL.
A partir de los valores de las dimensiones de la caja se encotr\'{o} una reducci\'{o}n en el coeficiente de difusi\'{o}n cuando se agrega un 15\% de estafiloxantina. Adicionalmente con los valores encontrados en la literatura se encontraron coeficientes de difusi\'{o}n en el orden de los 15 $(\mu mathrm{m})^2/s$ para los membranas puras o con una STX, mientras que para los sistemas con 15\% STX, la constante de difusi\'{o}n se reduce al rango de los  5 a 10 $(\mu mathrm{m})^2/s$.
Los perfiles de presi\'{o}n no cambian en su orden de magnitud (100 - 1000bar) entre un sistema y otro pero hay un aumento en el ruido entre los sistemas puros y los de una STX, con 15\% el ruido est\'{a} en un estado intermedio entre el sistema puro y el de la membrana con una estafiloxantina. 
El perfil de presi\'{o}n toma dos valles negativos debidos a interacciones atractivas en la interfaz, estos dos valles son caracter\'{i}sticos de la fase $L_{\alpha}$.
El hecho de que el estr\'{e}s sea mayoritariamente negativo indica que la membrana est\'{a} estresada y que hay una curvatura positiva.
