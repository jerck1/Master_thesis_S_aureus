\section{Orientación}
Si se define el ángulo de orientación de STX como el ángulo más pequeño entre el vector normal a la membrana y el vector que recorre la cadena diaponeuresporenoica (o poliénica), entonces el ángulo es de 0 \textdegree cuando la cadena de STX se encuentre en una orientación vertical mientras que el ángulo será de 90 \textdegree si  la cadena de STX se ncuentra en una orientación horizontal.\\
%resultados
%#orientación
Las distribuciones del ángulo de orientación de STX se encuentran en la figura FIGURA. El ángulo de orientación para una sola estafiloxantina refleja un rango más corto de valores (distribuci\'{o}n m\'{a}s angosta) que para el caso de 15\% de STX, además si se compara una STX inmersa en DMPG con una STX inmersa en DPPG se observa que el rango de posibles orientaciones de STX es mayor para el caso DMPG que para DPPG. De estas diferencias en los rangos se infiere que STX tiende a estar más vertical en presencia de DPPG que en presencia de DMPG. Otra característica que se encuentra en una de las distribuciones para una STX y no en la otra distribución es el hecho para DMPG solo hay un máximo, mientras que para DPPG hay dos máximos, es decir DPPG es una distribución bimodal. --Se pueden conjeturar varias cosas sobre la distribución bimodal, por ejemplo que esta se obtiene porque alguna parte de la molécula se puede mover con mayor facilidad para el caso de DPPG, alguna parte como la cola de la cadena poliénica de STX o el az\'{u}car de STX. Otra posibilidad es que exista alguna interacción que estabilice estas dos posibles orientaciones de STX.--\\

Para el caso de una sola molécula de STX y usando los parámetros obtenidos por optimización QM (STX rígida) se observa que disminuye la amplitud del segundo pico en DPPG, lo cual nos indica que la aparición de un rango más amplio de orientaciones de STX en DPPG se debía a una mayor flexibilidad de la cadena poliénica, con lo cual la molécula podía variar su longitud.\\

Para un 15\% de STX aparecen los dos picos tanto en el caso DMPG como en el caso DPPG y los rangos posibles son similares, pero la media para el caso de DMPG es mayor que para el caso de DPPG, esto nos indica que el ángulo tiende a ser más vertical a un 15\% de STX inmersa en DPPG.\\ 

Por otro lado, en los diagramas de dispersión de la posición del azúcar respecto al ángulo de orientación, se observa una ligera tendencia a la inserción del azúcar dentro de la membrana a medida que aumenta el ángulo de orientación. Esta característica desaparece para el caso 15\% STX en DPPG, en donde no se observa una correlación entre el ángulo y la posición del azúcar.\\
\section{Parámetro de Orden}
%#parámetro de orden
En la figura FIGURA se encuentra el parámetro de orden de los lípidos para todos los sistemas: Figura a): cadena 1 DMPG, b) cadena 2 DMPG, c): cadena 1 DPPG, d) cadena 2 DPPG. El parámetro de orden de los lípidos se incrementa significativamente  cuando la membrana contiene 15\% de STX llegando a ser este aumento de entre 0.02 a 0.04 cuando miramos los carbonos C4 a C6 de todos los casos, es decir, los lípidos de la membrana están más organizados cuando hay mayor contenido de STX . Además, si comparamos los casos de DMPG contra DPPG vemos que el parámetro de orden es menor en el orden de 0.01 entre los carbonos C4 a C6 para el caso DMPG que para el caso DPPG, entonces se puede ver que la membrana está más ordenada cuando el 15\% de STX está inmerso en DPPG. Otra diferencia se encuentra entre las cadenas 1 y 2 para cada caso. En ambos sistemas la cadena 2 presenta un valle en el átomo C2, esto posiblemente se deba a que una de las cadenas es más larga que la otra.\\

En la figura FIGURA se muestra el parámetro de orden de STX para la cadena acil de STX a): DMPG, b): DPPG, se escogió esta cadena ya que al ser más móvil esperamos que sea más sensible a cambios en las propiedades de los parámetros de la molécula. Para los casos de una sola STX no hubo un incremento significativo del parámetro de orden pero para un 15\% incrementó significativamente excepto para el primer átomo de la cadena acil. Para una sola molécula.
\section{Espesor de la Membrana}
%espesor de la membrana

En la figura FIGURA, se muestra la densidad electrónica normalizada respecto al eje z de la membrana (normal), las líneas punteadas son las posiciones de los fosfatos para las bicapas puras. Tomando la las líneas punteadas, el espesor de la membrana para DMPG es de 3.5nm, mientras que para DPPG es 3.8nm. La densidad electrónica de la cadena acil de los lípidos no es cero en el centro de la membrana, esto quiere decir que hay interdigitación. El oxígeno O3,  presente en el azúcar de STX, está más inmerso en la membrana puesto que su posición es menor a la del fosfato de la derecha, esto quiere decir que el azúcar está hacia adentro de la membrana. En todos los casos la posición de este O3, en promedio, está en el mismo lugar, sin embargo, en el caso de 15\% de STX. \\

El parámetro de orden está relacionado con las constantes elásticas del sistema REF DOULIEZ - ORDER PARAMETER.

\'{A}ngulo medio: 15STX-DMPG>15STX-DPPG \\
Espesor: DPPG=15STX-DPPG>15STX-DMPG=DMPG \\
APL: DMPG=DPPG>15STX-DMPG>15STX-DPPG \\
