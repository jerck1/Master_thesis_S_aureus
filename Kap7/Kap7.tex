\chapter{Conclusiones y recomendaciones}
\section{Conclusiones}
La estafiloxantina es un carotenoide estructuralmente estable (en conformaci\'{o}n  \textit{trans}) en la regi\'{o}n de sus enlaces dobles conjugados, unida en un extremo a un az\'{u}car ligeramente inmerso en la membrana y en otro extremo a un ``gancho'' con altos grados de libertad de movimiento, tal como se ha comprobado con las  distribuciones de \'{a}ngulos dihedros. Adem\'{a}s las diferencias presentadas de los \'{a}ngulos dihedros entre los par\'{a}metros de CHARMM y por QM no son notorias. La estafiloxafiloxantina tiende a acomodarse seg\'{u}n el espesor de la membrana tomando una orientaci\'{o}n inclinada o mas vertical. Si est\'{a} inmersa en una membrana de menor grosor, como lo es DMPG, tiende a ser m\'{a}s horizontal, si est\'{a} inmersa en una membrana de mayor grosor, como lo es DPPG tiende a estar m\'{a}s vertical y puede muestrear un rango m\'{a}s amplio de orientaciones, generando una distribuci\'{o}n bimodal. Es importante notar, que el modelo que se segu\'{i}a anterior a estas simulaciones propon\'{i}a una orientaci\'{o}n puramente vertical para estafiloxantina, aunque ya para otros carotenoides se hab\'{i}a encontrado por simulaciones de din\'{a}mica molecular una orientaci\'{o}n angulada. La orientaci\'{o}n angulada de la mol\'{e}cula puede tener implicaciones estructurales importantes ya que implica la presencia de una red horizontal de carotenoides en la membrana. Es interesante observar el comportamiento bimodal en DPPG, en donde se detecta un segundo pico en la distribuci\'{o}n para conformaciones m\'{a}s verticales de la mol\'{e}cula. El aumento de grosor de la membrana parece acomodar mejor una conformaci\'{o}n vertical permitiendo esta segunda orientaci\'{o}n. \\

En cuanto al efecto de la estafiloxantina sobre la membrana es posible ver que el estudio de las propiedades locales de la membrana permite dilucidar ciertos efectos que no son visibles con el uso exclusivo de las propiedades globales. Es as\'{i} como se ve que al agregar un 15\% de estafiloxantina, de acuerdo a los resultados de g\_lomepro, la membrana se vuelve m\'{a}s ordenada y m\'{a}s compacta lateralmente. El compactamiento lateral se puede explicar por una disminuci\'{o}n del \'{a}rea por l\'{i}pido en las regiones en las que se encuentra la estafiloxantina Esto probablemente se debe a la peque\~{n}a cabeza polar de la estafiloxantina (glucosa), comparado con la cabeza polar de los l\'{i}pidos PG (fosfatidilgliceroles). La reducci\'{o}n local de \'{a}rea debido a la presencia de estafiloxantina es un efecto que solo se ve al analizar a nivel local la membrana, aunque en promedio la disminuci\'{o}n no es significativa, esto se debe a que el \'{a}rea por l\'{i}pido presenta distribuciones bimodales m\'{a}s anchas que en las membranas puras.\\

Los cambios en la geometr\'{i}a de la membrana vienen aunados a una disminuci\'{o}n en el coeficiente de difusi\'{o}n para composiciones del 15\% de estafiloxantina. Estas disminuciones se ven porque los coeficientes de difusi\'{o}n para la caja finita van del orden de los 15 $(\mu \mathrm{m})^2/s$ para los membranas puras o con una STX, y disminuyen para los sistemas con 15\% STX, al rango de los  5 a 10 $(\mu \mathrm{m})^2/s$. Se ha demostrado en otras bacterias que estas reducciones en las constantes de difusi\'{o}n resultan en cambios en los procesos respiratorios bacterianos al cambiar la din\'{a}mica de las reacciones asociadas al transporte de electrones en la membrana.\\

Por otro lado los resultados de gromacs muestran que el espesor de la membrana no disminuy\'{o} significativamente, si solo se observa la posici\'{o}n de los grupos fosfato. Esto puede ocurrir por varias causas: una es la interdigitaci\'{o}n de los carotenoides y de los l\'{i}pidos que permite acomodar las cadenas en las dos monocapas, otra es debido a  la capacidad de la estafiloxantina de muestrear diversas orientaciones, donde el rango de orientaciones es mayor en cuanto menor sea la longitud de la cadena acil lip\'{i}dica, lo cual permite acomodar la mol\'{e}cula sin implicar un cambio en el grosor de la membrana.\\
%Este resultado contrasta con los resultados de g\_lomepro donde el espesor de la membrana disminuye significativamente al agregar un 15\% de estafiloxantina.\\

Otra propiedad que podr\'{i}a dar cuenta del efecto de la estafiloxantina en la membrana es el perfil de presi\'{o}n aunque no se hayan encontrado cambios significativos en orden de magnitud (de $100\mathrm{bar}-1000\mathrm{bar}$) entre los picos y los valles de los perfiles de presi\'{o}n al incrementarse la concentraci\'{o}n de estafiloxantina. Entre los cambios presentados est\'{a} que al agregar una sola estafiloxantina, el perfil de presi\'{o}n presenta m\'{a}s ruido, en cambio cuando la concentraci\'{o}n de estafiloxantina es de 15\% se vuelve a estabilizar el perfil de presi\'{o}n. En cuanto a la forma de la curva se encontr\'{o} que el perfil de presi\'{o}n presentaba dos valles negativos debido a interacciones atractivas en la interfaz, estos dos valles son caracter\'{i}sticos de la fase l\'{i}quida desordenada $L_{\alpha}$ y representan las interacciones hidrof\'{o}bicas de las cadenas apolares. Adem\'{a}s el hecho de que el estr\'{e}s sea mayoritariamente negativo indica que la membrana presenta tensi\'{o}n superficial y que hay una curvatura netamente positiva.\\
%Las conclusiones constituyen un cap\'{\i}tulo independiente y presentan, en forma l\'{o}gica, los resultados de la tesis  o trabajo de investigaci\'{o}n. Las conclusiones deben ser la respuesta a los objetivos o prop\'{o}sitos planteados. Se deben titular con la palabra conclusiones en el mismo formato de los t\'{\i}tulos de los cap\'{\i}tulos anteriores (T\'{\i}tulos primer nivel), precedida por el numeral correspondiente (seg\'{u}n la presente plantilla).\\

\section{Recomendaciones}
Respecto al perfil de estr\'{e}s se recomienda imprimir el archivo .trr con un menor paso del tiempo, esto permitir\'{a} un mayor muestreo para el c\'{a}lculo del estr\'{e}s y posiblemente disminuir\'{a} el ruido mostrado en las gr\'{a}ficas de perfiles de estr\'{e}s. Se debe analizar el efecto de la estrategia utilizada para suavizar los perfiles para entender su validez.\\

Tambi\'{e}n es importante hacer una validaci\'{o}n para saber si los perfiles de estr\'{e}s son consistentes, esto se podr\'{i}a hacer hallando la m\'{o}dulo de elasticidad, de esta manera podr\'{i}a comprobarse si estos valores est\'{a}n dentro de los valores reportados por \cite{Perez-Lopez2019VariationsProperties}.\\

En cuanto al coeficiente de difusi\'{o}n es importante revisar la viscosidad ya que esta propiedad normalmente no ha sido medida experimentalmente para los sistemas de membranas que se est\'{a}n trabajando.\\

En cuanto al espesor global es recomendable hallar un espesor de la membrana pesando tanto los grupos fosfato de la membrana como los az\'{u}cares de la estafiloxantina.\\

%Se presentan como una serie de aspectos que se podr\'{\i}an realizar en un futuro para emprender investigaciones similares o fortalecer la investigaci\'{o}n realizada. Deben contemplar las perspectivas de la investigaci\'{o}n, las cuales son sugerencias, proyecciones o alternativas que se presentan para modificar, cambiar o incidir sobre una situaci\'{o}n espec\'{\i}fica o una problem\'{a}tica encontrada. Pueden presentarse como un texto con caracter\'{\i}sticas argumentativas, resultado de una reflexi\'{o}n acerca de la tesis o trabajo de investigaci\'{o}n.\\
